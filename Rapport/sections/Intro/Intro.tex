\documentclass[../../main.tex]{subfiles}
\begin{document}
\section{Midway Report}

\subsection{Problem Statement}
I will attempt to create a clustering algorithm, that can compete with, or even surpass, uClust in terms of speed, without losing precision, when working with a dataset of 500000 RNA/DNA sequences of length 500-1500.

\subsection{Analysis}
The very high number of sequences that have to be compared means that the speed of the algorithm that i develop is extraordinary. uClust is at the moment number one contender in terms of speed without loss of much precision.\\

Studies have shown that the implementation of Minhash in \cite{MinhashMapreduce} have approached both the speed and precision of uClust. This method is quite different from the centroid based approach of uClust, and therefore spurred my interest.\\

My attempt will be to apply some improvements to the Minhash implementation above, and see whether i can achieve the same speed and precision as that of uClust. This implementation i will call Max-Minhash.

\subsection{Status}

Currently i have finished the following tasks:
\begin{enumerate}
\item Finished a prototype of my Max-Minhash algorithm. It has the following functionalities
\begin{enumerate}
\item Singlethreaded implementation of a Max-Minhash algorithm
\item A hashfunction based on the one in \cite{MinhashMapreduce}
\item Reads sequences from .seq files. 
\end{enumerate}
\item Finished a draft of the following two sections: Universal Hashing and Minhash
\end{enumerate}

What is left to be done are the following tasks

\begin{enumerate}
\item A comprehensive test of several universal hashing schemes to find a method that might be faster than the currently used one.
\item Developing the prototype so that it has the following improvements
\begin{enumerate}
\item The newfound quickest hashfunction.
\item Programmed to work in a Mapreduce framework, so that it can be run in Hadoop
\item Able to read .fasta files. Has to be able to handle sequences that contain the character N.
\end{enumerate}
\item Comprehensive tests of my algorithm in comparison to the uSearch algorithm, to see which is faster and most precise
\item As my algorithm uses K-mers as the basis for its Minhash input, a comprehensive test of which k-mer size is the most optimal for precision and speed.
\item Documenting all the tests that i perform above. 
\item Writing a conclusion.
\item Improving the sections i wrote about universal hashing, and the Minhash section
\end{enumerate}

It should be noted that quite a few of these tasks can be completed without having 2.b ready, read. 1, 2.c, and 4. These will therefore be done parallel to implementing the Mapreduce framework, some using the current prototype.

\subsection{Time plan}
I will estimate each of the above remaining tasks and insert them into a Gantz diagram.\\

\begin{enumerate}
\item Test of universal hashing schemes.
\begin{enumerate}
\item {\bf Product:} Imlpementation of, and a comprehensive test of several hashing schemes. Includes graphs and data. Includes conclusivee findings.
\item {\bf Resources:} Java, .seq test data
\item {\bf Dependencies:} Section on Universal hashing
\item {\bf Workload:} 2 days
\end{enumerate}
\item Implement IO handler that can read Fasta files
\begin{enumerate}
\item {\bf Product:} Implementing IO handler that can read fasta files.
\item {Resources:} Java, fasta files
\item {Dependencies:} Defining how to handle sequences containing the character N, Java
\item {\bf Workload:} 1 day
\end{enumerate}
\item Mapreduce Framework
\begin{enumerate}
\item {\bf Product:} Improving the prototype to be runnable in a Mapreduce Framework.
\item {\bf Resources:} Java, Hadoop
\item {\bf Dependencies:} Knowledge of Hadoop, Researching Mapreduce
\item {\bf Workload:} 10 days
\end{enumerate}
\item Test of  K-mer sizes
\begin{enumerate}
\item {\bf Product:} Documented tests of k-mer sizes to find the appropriate k-mer size for RNA/DNA sequences. Includes data and graphs of results.
\item {\bf Resources:} Java, .seq files
\item {\bf Dependencies:} A Gold standard, Java, .seq test data
\item {\bf Workload:} 2 days
\end{enumerate}
\item Final comparative test
\begin{enumerate}
\item {\bf Product:} Documented final comparative test of Max-min implementation alongside uClust. Includes data and graphs of results.
\item {\bf Resources:} Java, Hadoop, usearch
\item {\bf Dependencies:} A finished implementation of Max-min, A sorted .fasta file for smallmem usearch.
\item {\bf Workload:} 3 days.
\end{enumerate}
\item Section about my algorithm
\begin{enumerate}
\item {\bf Product:} Writing a section describing my algorithm. Includes examples from code, etc.
\item {\bf Resources:} Latex
\item {\bf Dependencies:} Final implementation of Max-Min hash algorithm.
\item {\bf Workload:} 1 day.
\end{enumerate}
\item Finetuning the report.
\begin{enumerate}
\item {\bf Product:} Writing the conclusion and finetuning the other sections. Making a red thread between all sections. Assuring the scientific correctness of introductory sections about minhash and universal hashing.
\item {\bf Resources:} Latex, Articles about previous sections, feedback
\item {\bf Dependencies:}  Section about algorithm, Section about Minhash, Section about universal hashing, The documented tests described above. 
\item {\bf Workload:} 10 days.
\end{enumerate}
\end{enumerate}

The timeplan will be describe so that the number above will be completed within the week it is set into:\\

\noindent Week 15: (1), (2)\\
Week 16: (3), (4)\\
Week 17 - 19: (3),(6)\\
Week 20: (5)\\
Week 21-23: (7)


 
 
\end{document}

